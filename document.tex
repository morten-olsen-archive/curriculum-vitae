\begin{luacode}
require("lualibs.lua")
local file = io.open('data.json')
local jsonstring = file:read('*a')
file.close()
jsondata =  utilities.json.tolua(jsonstring)
\end{luacode}

\begin{luacode}
	tex.print("\\begin{cvtitle}{Morten Olsen}")
	for key, value in pairs(jsondata["info"]) do
		tex.print("\\cvinfo{" .. value["name"] .. "}{" .. value["value"] .. "}")
	end
	tex.print("\\end{cvtitle}")
\end{luacode}

\begin{luacode}
	tex.print("\\begin{columns}")
	tex.print("\\section*{Who am i?}")
	tex.print("\\begin{markdown}")
	tex.print(jsondata["aboutMe"])
	tex.print("\\end{markdown}")
	tex.print("\\end{columns}")
\end{luacode}

\section*{Platforms and Languages}
	List of platforms which I have worked with and knows about. The list is a shortend down version with only the most known entries included.\\\\

\begin{cvskills}
\begin{luacode}
for key, value in pairs(jsondata["skills"]) do
	tex.print('\\cvskill{' .. value["title"] .. '}{' .. value["level"] .. '}')
end
\end{luacode}
\end{cvskills}

\section*{Experiences}

\begin{luacode}
for key, value in pairs(jsondata["positions"]) do
	tex.print("\\begin{cvexp}{" .. value["company"]["name"] .. "}{" .. value["startDate"] .. "}{" .. value["endDate"] .. "}{" .. value["title"] .. "}")
	tex.print("\\begin{markdown}")
	tex.print(value["description"])
	tex.print("\\end{markdown}")
	tex.print("\\end{cvexp}")
end
\end{luacode}

\section*{Frameworks}

This list of frameworks are a currated list of frameworks I have been using recently, and therefore are all amongst frameworks which i prefere to work with.\\

\begin{cvskills}
	\cvskill{React}{5}
	\cvskill{Redux}{5}
	\cvskill{Webpack}{5}
	\cvskill{RxJS}{5}
	\cvskill{RxDB}{5}
	\cvskill{Styled-Components}{5}
	\cvskill{MVC}{5}
	\cvskill{Web API}{5}
	\cvskill{Babel}{5}
\end{cvskills}

\section*{Projects of interest}
A list of projects I have worked with or am working on, for which the source is publicly available. Keep in mind that some of these projects are in early stages, and some are simply created for the training, and may not represent the the actual way i think production code should be structured\\
\hrule

\begin{luacode}
for key, value in pairs(jsondata["projects"]) do
	tex.print("\\begin{cvproj}{" .. value["name"] .. "}{" .. value["url"] .. "}{" .. value["tagline"] .. "}")
	tex.print("\\begin{markdown}")
	tex.print(value["description"])
	tex.print("\\end{markdown}")
	tex.print("\\end{cvproj}")
end
\end{luacode}

\begin{cvproj}{Curriculum Vitae}{github.com/morten-olsen/curriculum-vitae}{Automated CV building}
Another small fun project is actually this resume, as it is created in \LaTeX, versioned using \textit{Git} on \textit{Github} and set to automatically build and create multible release versions using \textit{Travis CI} and \textit{Docker}
\end{cvproj}
%%\begin{card}
%%	\begin{minipage}{1.6in}
%%		\cardline{{\Large Morten Olsen}}
%%		\cardline{Frontend Technical Lead}
%%		\cardline{Trendsales ApS}
%%		\cardline{}
%%		\cardline{morten@olsen.io}
%%		\cardline{github.com/morten-olsen}
%%		\cardline{+45 31 24 34 14}
%%	\end{minipage}
%%	\hfill
%%	\begin{minipage}{1.3in}
%%		\includegraphics[height=1.3in]{me.jpg}
%%	\end{minipage}
%%\end{card}

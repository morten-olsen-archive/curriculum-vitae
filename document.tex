\begin{cvtitle}{Morten Olsen}
	\cvinfo{Email}{morten@olsen.io}
	\cvinfo{Phone}{+45 31 24 34 14}
	\cvinfo{Location}{Copenhagen, DK}
	\cvinfo{Github}{https://github.com/morten-olsen}
	\cvinfo{CodePen}{https://codepen.io/morten-olsen}
\end{cvtitle}

\begin{columns}
	\section*{Who am i?}
	Hell bend on a conquest to take over the digital world (and the physical, should the chance arise). Am I over ambitious? perhabs, but with a proven track record in most aspects of things which can process 1s and 0s and a mind which runs at a speed which can battle a well trained race horse, I believe I am equipt to undertake this voyage!
\end{columns}

\section*{Platforms and Languages}
	List of platforms which I have worked with and knows about. The list is a shortend down version with only the most known entries included.\\\\
\begin{cvskills}
	\cvskill{C\#}{5}
	\cvskill{JavaScript}{5}
	\cvskill{NodeJS}{5}
	\cvskill{HTML}{5}
	\cvskill{CSS}{5}
	\cvskill{TypeScript}{5}
	\cvskill{PHP}{2}
	\cvskill{Docker}{4}
	\cvskill{T-SQL}{3}
	\cvskill{MySQL}{3}
	\cvskill{Postgres}{2}
	\cvskill{Elastic Search}{3}
	\cvskill{GraphQL}{4}
	\cvskill{Linux}{5}
	\cvskill{Nginx}{3}
\end{cvskills}

\section*{Experiences}

\begin{cvexp}{Trendsales}{2016}{Current}{Frontend Technical Lead}
	In 2015 Trendsales decided to build an entirely new platform, where I was tasked with the creation of a new frontend architecture. The work began in 2016 with just me on the project and consisted of a proof of concept version containing everything from framework selection, structuring, style guides build chain, continous deployment and an actual initial working version. This is the platform I have been given technical ownership over and which I, along with 2 others, has been hard at work expanding over the past year. Currently the platform is powering \textit{m.trendsales.dk} and hopefully at the end of 2017 it will have taken over the entire \textit{www.trendsales.dk} domain. The project is build using React and state managment are done using Redux. In addition to the ofter shelve frameworks, we have also needed to develop quite a few bespoke frameworks, in order to meet demands, amongst the challenges those where created to overcome are:
	\begin{itemize}  
		\item Introducing a new navigational paradime
		\item Create a more flexible routing mechanism
		\item Be able to serve skeleton page, for page transitions while still being able to create complete server side pages
		\item Ensure project flows between multiple systems such as Github, Jira, Octopus Delpoy, AppVeyor and Docker
	\end{itemize} 
\end{cvexp}

\begin{cvexp}{Trendsales}{2012}{2015}{iOS and Android Developer}
 I was offered the responsibility for the iOS platform, which was a task that required a new app to be build from the ground up and had to be so using \textit{Xamarin\footnote{Called MonoTouch back then}}. In addition to that, a new API to support the app along with support for our larger vendors was needed which had to be build using something closely simular to \textit{Microsoft MVC} so that other people could join the project at a later state.
 
 The project started in October with the initial version available to our users in late December.
 This project represent my first adventure into mobile development, and became an app with more than 15.000.000 screen views and 1.500.000 sessions pr month.
 
 After that, I joined two other colleagues, whom where working on an Android version of the app, to form a join mobile development team. 
\end{cvexp}

\begin{cvexp}{Trendsales}{2012}{2012}{Web Developer}
	I was offered a part time job at Trendsales, where my main responsibility where maintaining the API which powered the iOS app. Quickly my tasks became more diverse and I ended using about 25-50\% of my time on the API, while the remaining was spend doing work on the platform in general.
\end{cvexp}

\begin{cvexp}{BilZonen}{2010}{2012}{Web Developer}
	I work as a part time web developer on bilzonen.dk . I have worked with both day-to-day maintenance and large scale projects (new search module, integration of new data catalog, mobile site, new-car-catalog and the entire dealer solution). The page is a Umbraco solution, with all .NET (C\#) code. I have introduced a new custom build provider-model system, which allows data-providers to move data between data stores, external services and the site. (search, caching and external car date is running through the provider system). In addition i have set up the development environment , from setting up virtual server hosts to building custom tool for building and unit testing.
\end{cvexp}

\begin{cvexp}{Haastrup IT}{2009}{2010}{System Developer}
	I have worked as a part time project koordinator and systems developer, sitting with responsibility for a wide variety of projects including projects for "Københavns Kommune" (Navision reporting software) and "Syddanmarks kommune" (Electronic application processing system). Most projects were made in C\#, but also PHP, VB, ActionScript. In addtion to that i maintained the in-house hosting setup.
\end{cvexp}

\begin{cvexp}{Sydbank}{2007}{2009}{IT Hotline}
	I work as part time supporter to customers (private and business) and staff, on Sydbanks different electronic banking systems. Mostly telephonic bug finding and PC setup.
\end{cvexp}

\section*{Frameworks}

This list of frameworks are a currated list of frameworks I have been using recently, and therefore are all amongst frameworks which i prefere to work with.\\

\begin{cvskills}
	\cvskill{React}{5}
	\cvskill{Redux}{5}
	\cvskill{Webpack}{5}
	\cvskill{RxJS}{5}
	\cvskill{RxDB}{5}
	\cvskill{Styled-Components}{5}
	\cvskill{MVC}{5}
	\cvskill{Web API}{5}
	\cvskill{Babel}{5}
\end{cvskills}

\section*{Projects of interest}
A list of projects I have worked with or am working on, for which the source is publicly available. Keep in mind that some of these projects are in early stages, and some are simply created for the training, and may not represent the the actual way i think production code should be structured\\
\hrule

\begin{cvproj}{LifeFlow Mind}{github.com/lifeflow-mind}{Evernote meets \LaTeX meets Jupyter}
	I love Evernote easy digital not storage, but as a developer and general nerd, its features often comes up short, which is why I am working on a project named Mind, which attempts to bridge the best aspect of Evernote, with a dynamic view system, so new views (for instance a calendar or a contact view) can be introduced. In addition, it has the capability to render mathematical formulas using \LaTeX and execute code snippets, to test and present code directly inside documents. A somewhat working demo can be found at https://morten-olsen.github.io/mind
\end{cvproj}

\begin{cvproj}{Redux App State}{github.com/trendsales/redux-app-state}{Advanced app like navigation on the web}
	One of the biggest issues I had to overcome when I began building the revamped Trendsales front end was that the design called for an advanced navigation strategy, which did not fit into the webs linary document structure, but rather a branched multi layer navigation. Therefore i created this project to create a high level abstraction using a navigation trap to allow for much more advanced navigation patterns
\end{cvproj}

\begin{cvproj}{Snipsy}{github.com/morten-olsen/snipsy}{Snippet installation made easy}
	One recurent issue when dealing with fast responding web pages is the issue of external resources, since these often comes with a short cache validity and issues such as missing compression. To improve this, I am working on a system, where installation into a platform can be offered by the external part, so for instance, \textit{Google Tag Manager} can be installed using a Google build installer to ensure it gets setup correct. After that the snippet lives inside the application, where it caches and serves these otherwise external resources.
\end{cvproj}

\begin{cvproj}{Labyrinth Guards}{github.com/morten-olsen/labyrinth-guards}{A/B testing in React}
	As we need to do a lot of A/B testing at Trendsales and no real solutions with the level of flexibility we needed where avalable for React, I created \textit{Labyrinth Guard\footnote{The name is a reference to https://xkcd.com/246/}} which support server side caching of test suites, a high level of flexibility and easy test result gatherings. Labyrinth Guards in itself is only a framework, which uses external services for test result gathering, currently supporting Google Analytics, but another standalone server is in the making.
\end{cvproj}

\begin{cvproj}{wolfsquad.co}{github.com/morten-olsen/wolfsquad}{I fun experiment with encryption}
	This is an old project I did to play around with encryption in JavaScript. The initial boot loaded is served as regular JavaScript, but after login, all communication has a level of encryption applied based upon the password, which is never send the the server, as its encryption abilities offer this feature, without having to transfer sensitive information. It is currently available at https://http://wolfsquad.co/ with the username \textit{admin} and the password \textit{password}
\end{cvproj}

\begin{cvproj}{Curriculum Vitae}{github.com/morten-olsen/curriculum-vitae}{Automated CV building}
Another small fun project is actually this resume, as it is created in \LaTeX, versioned using \textit{Git} on \textit{Github} and set to automatically build and create multible release versions using \textit{Travis CI} and \textit{Docker}
\end{cvproj}
\begin{card}
	\begin{minipage}{1.6in}
		\cardline{{\Large Morten Olsen}}
		\cardline{Frontend Technical Lead}
		\cardline{Trendsales ApS}
		\cardline{}
		\cardline{morten@olsen.io}
		\cardline{github.com/morten-olsen}
		\cardline{+45 31 24 34 14}
	\end{minipage}
	\hfill
	\begin{minipage}{1.3in}
		\includegraphics[height=1.3in]{me.jpg}
	\end{minipage}
\end{card}
